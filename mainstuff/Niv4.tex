\chapter*{Niveau 4}
\addcontentsline{toc}{chapter}{Niveau 4}

Du går hvor lyset ikke når og hvor mørket ikke tør. Dine skridt høres aldrig og din hånd er stille. Dinne venner frygter intet, men dine fjender sover aldrig. Du er skyggerne selv.


Som \textbf{Snigmorder} er et liv noget du tager på regulær basis. Din kniv drypper med blod og dine fjender ved sig aldrig sikre for hvornår din pil eller gift finder dem.\\

En \textbf{Mestertyv} ses aldrig. Det eneste bevis på at du har været i et område er at alt af værdi er forsvundet. Folk siger at du kan stjæle alt hvad der ikke er boltet fast, men du ved at det ikke er rigtigt... En bolt kan trods alt løsnes med det rigtige værktøj.\\

\textbf{Du skal vælge dit speciale med omhug, da dette ikke kan ændres efter dette er valgt.}\\

\begin{tabular}{|p{0.3525\textwidth}|p{0.1175\textwidth}|p{0.3525\textwidth}|p{0.1175\textwidth}|}
\hline
\rowcolor{cerulean!80}
 \multicolumn{2}{|c|}{  Snigmorder } & \multicolumn{2}{|c|}{ Mestertyv }\\
\hline
\rowcolor{cerulean!40}
    Evne navn & Pris i XP & Evne navn & Pris i XP\\ \hline
    Alkymi Niv. 1 \& 2 & 3 & Dirke låse Niv. 3 & 2 \\ \hline
    Lejekontrakt & 2 & Hurtige fingre & 3\\\hline
    Mesterbueskytte & 4 & Lange fingre & 3\\\hline
    Snigmord Niv. 2 & 3 & Lommetyv Niv. 3 & 2\\
\hline
\end{tabular}

\section*{Evne beskrivelse for Snigmorder}
\addcontentsline{toc}{section}{Evne beskrivelse for Snigmorder}

\subsection*{Alkymi Niv. 1 \& 2}
\addcontentsline{toc}{subsection}{Alkymi Niv. 1 \& 2}
Med evnen Alkymi kan du lave drikke, der påvirker spillet og spillerne. Disse drikke er ikke magiske, men kemiske.\\
Hvis du tager denne evne er det meget vigtigt, at du læser sektionerne: Drikke, Opskrifter og Ingredienser.

\subsection*{Alkymi Niv. 1}
\addcontentsline{toc}{subsection}{Alkymi Niv. 1}
Du kan nu brygge en af de nedenstående drikke. Alle disse drikke kræver specifikke ingredienser, og kun du kan blande dem korrekt.\\

\begin{table}[H]
    \centering
    \begin{tabular}{|c|c|}
        \rowcolor{cerulean!80}\hline
        Drik & Gift \\\hline
        Enkens tåre &  Dværge snaps \\\hline
        Naturens bryg & Nervegift \\\hline
        Sten næver &  \\\hline
    \end{tabular}
    \caption{Oversigt over niveau 1 drikke.}
\end{table}

\begin{gift*}[Dværge snaps]
\textbf{Type:} Negativ Gift\\
\textbf{Ingredienser:} Øl fra kroen \& Vorterod\\
\textbf{Indtagelse:} Drikkes\\
\textbf{Effekt:} Ved indtagelse bliver man øjeblikkeligt meget beruset. Virkningen forsvinder, hvis spilleren mister bevidstheden eller 15 min efter indtagelse. Drikke efterlader ingen tømmermænd.\\
\end{gift*}

\begin{drik*}[Enkens tåre]
\textbf{Type:} Øjeblikkelig Drik\\
\textbf{Ingredienser:} Kongetand \& rent vand\\
\textbf{Indtagelse:} Drikkes\\
\textbf{Effekt:} Denne drik fjerne alle effekter af alkohol og gør øjeblikkeligt spilleren ædru. Dette betyder at de ikke vil være påvirket af alkohollet eller evner der kræver man er under effekten for alkohol.\\
\end{drik*}

\begin{drik*}[Naturens bryg]
    \textbf{Type:} Øjeblikkelig Drik\\
    \textbf{Ingredienser:} Ingen\\
    \textbf{Indtagelse:} Drikkes\\
    \textbf{Effekt:} Denne drik helbreder 1 LP. Du kan aldrig blive helbredt for mere end dit max LP.\\
    \textbf{Særligt:} Denne drik kan brygges hver anden time, men vare kun til slutningen af scenariet.
\end{drik*}

\begin{gift*}[Nervegift]
    \textbf{Type:} Øjeblikkelig Gift\\
    \textbf{Ingredienser:} Ingen\\
    \textbf{Indtagelse:} Drikkes / Blandes med væske\\
    \textbf{Effekt:}Denne drik skader 1 LP, og kan blandes i andre drikke uden, at den mister sin virkning.\\
   \textbf{Særligt:} Denne drik kan brygges hver anden time, men vare kun til slutningen af scenariet.
\end{gift*}

\begin{drik*}[Sten næver]
\textbf{Type:} Positiv drik\\
\textbf{Ingredienser:} 1 Dværgerod.\\
\textbf{Indtagelse:} Drikkes\\
\textbf{Effekt:} Når du drikker denne drik, får du +2NK i 30 min.
\end{drik*}


\subsection*{Alkymi Niv. 2}
\addcontentsline{toc}{subsection}{Alkymi Niv. 2}
Du kan nu brygge en af de nedenstående drikke ud over de drikke du har adgang til i niveau 1. Alle disse drikke kræver specifikke ingredienser, og kun du kan blande dem korrekt.\\

\begin{table}[H]
    \centering
    \begin{tabular}{|c|c|}
        \rowcolor{cerulean!80}\hline
        Drik & Gift \\\hline
        Blodorkens styrke &  Den lærtes mund \\\hline
        Drømmeløs hvile & Ewens humør \\\hline
        Helbredelsesdrik & Smertedrik \\\hline
        Kranieforstærker & Svag Giftdrik\\\hline
        Stenhud &  Søvndrik\\\hline
        \multicolumn{2}{|c|}{Speciel} \\\hline
        \multicolumn{2}{|c|}{Velsignet Olie} \\\hline
    \end{tabular}
    \caption{Oversigt over niveau 2 drikke.}
\end{table}

\begin{drik*}[Blodorkens styrke]
\textbf{Type:} Positiv drik \\
\textbf{Ingredienser:} 1 Dværgerod \& blod fra en blodork.\\
\textbf{Indtagelse:} Drikkes\\
\textbf{Effekt:} Når du drikker denne drik, får du +4NK i 30 min.
\end{drik*}

\begin{gift*}[Den lærtes mund]
\textbf{Type:} Negativ drik\\
\textbf{Ingredienser:} 3 Ildgræs\\
\textbf{Indtagelse:} Drikkes / Blandes med væske / Evnen Påfør Gift\\
\textbf{Effekt:} Den som indtager denne gift skal sige alt hvad de tænker, i de næste 10 minutter.
\end{gift*}

\begin{drik*}[Drømmeløs hvile]
\textbf{Type:} Positiv Drik\\
\textbf{Ingredienser:} Blodbær \& Kongetand \& Tudsebark \\
\textbf{Indtagelse:} Drikkes\\
\textbf{Effekt:} Når denne drik indtages, falder spilleren i søvn i 10 min. Efter disse 10 min vil alle naturlige LP være genvundet, men vækkes spilleren inden 10 min er gået, vil drikken ikke have nogen effekt og spilleren genvinder ingen LP.\\
\end{drik*}

\begin{gift*}[Ewens humør]
\textbf{Type:} Negativ drik\\
\textbf{Ingredienser:} Tudsebark \& Ildgræs\\
\textbf{Indtagelse:} Drikkes / Blandes med væske\\
\textbf{Effekt:} Når du drikker denne drik, er du venner med alt og alle og har bare lyst til at snakke med alle. Denne drik vare i 30 minutters.
\end{gift*}

\begin{drik*}[Helbredelsesdrik]
\textbf{Type:} Øjeblikkelig Drik\\
\textbf{Ingredienser:} 2 Kongetand\\
\textbf{Indtagelse:} Drikkes\\
\textbf{Effekt:} Denne drik helbreder 2 LP. Du kan aldrig blive helbredt for mere end dit max LP.\\
\end{drik*}


\begin{drik*}[Kranieforstærker]
\textbf{Type:} Positiv Drik\\
\textbf{Ingredienser:} Ildgræs \& Jern\\
\textbf{Indtagelse:} Drikkes\\
\textbf{Effekt:} Når du drikker denne drik bliver du immun overfor evnen Bonk de næste 30 min.\\
\end{drik*}

\begin{drik*}[Præstindens tårer]
\textbf{Type:} Øjeblikkelig Drik\\
\textbf{Ingredienser:} Velsignet vand \& Ildgræs \& Tudsebark \& Kongetand.\\
\textbf{Indtagelse:} Drikkes\\
\textbf{Effekt:} Fjerner alle positive og negative effekter fra Drikke og Gifte.
\end{drik*}

\begin{gift*}[Smertedrik]
\textbf{Type:} Negativ Gift\\
\textbf{Ingredienser:} 2 Vorterod\\
\textbf{Indtagelse:} Drikkes / Blandes med væske / Evnen Påfør Gift\\
\textbf{Effekt:} Den som indtager denne drik vil blive påvirket af smerte. Hvilket er en effekt, hvor du bliver ramt af en ulidelig smerte. Du skal derfor skrige og spille på, at du har ubærlige smerter i 30 sekunder.\\
\end{gift*}

\begin{drik*}[Stenhud]
\textbf{Type:} Positiv Drik\\
\textbf{Ingredienser:} Blod \& Kvælertang\\
\textbf{Indtagelse:} Drikkes\\
\textbf{Effekt:} Denne drik giver dig 2 LP over dit maks LP, som ikke kan genvindes. Effekten aftager, når der er gået 30 min eller, når du har mistet de ekstra LP.\\
\end{drik*}


\begin{gift*}[Svag Giftdrik]
\textbf{Type:} Øjeblikkelig Gift\\
\textbf{Ingredienser:} Tudsebark \& Vorterod\\
\textbf{Indtagelse:} Drikkes / Blandes med væske / Evnen Påfør Gift\\
\textbf{Effekt:} Når du bliver forgiftet af denne drik, mister du 2 LP.\\
\end{gift*}

\begin{gift*}[Søvndrik]
\textbf{Type:} Negativ Gift\\
\textbf{Ingredienser:} 2 Kvælertang\\
\textbf{Indtagelse:} Drikkes / Blandes med væske / Evnen Påfør gift.\\
\textbf{Effekt:} Når du indtager denne drik, vil du falde i søvn og sove de næste 5 min.\\
\end{gift*}

\begin{særlig*}
\textbf{Type:} Særlig\\
\textbf{Ingredienser:} Velsignet vand, en helbredende drik, Blodbær.\\
\textbf{Indtagelse:} Smøres på en klinge.\\
\textbf{Effekt:} Klingen denne olie smøres på vil give helligskade de næste 30 minutter. Dette skal markeres med et grønt bånd.
\end{særlig*}

\subsection*{Lejekontrakt}
\addcontentsline{toc}{subsection}{Lejekontrakt}
Snigmorderne kan spørge det sorte marked efter en lejekontrakt. Hvis kontrakten
bliver udført, tjener Snigmorderen Fjend\footnote{Møntfoden i A'kastin} eller anden værdier.\\

\subsection*{Mesterbueskytte}
\addcontentsline{toc}{subsection}{Mesterbueskytte}
Tyven skal trække en pil uset, sigte på sit offer i 15 sekunder og ramme offeret i ryggen. Hvis pilen rammer, skal han råbe “Baghold Død!”. Hvorefter offeret vil gå på 0 LP, uanset LP og RP.\\

\subsection*{Snigmord Niv. 2}
\addcontentsline{toc}{subsection}{Snigmord Niv. 2}
Du stikker en ubevæbnet person ned bagfra. Spilleren skal stå med ryggen til dig. Dette skader 5 og går igennem spillerens rustning.\\ 
En kniv føres under armen på målet og kommandoen: ”Baghold 6 i skade gennem rustning” siges, så målet kan høre det. Denne evne kan ikke bruges med to knive på samme tid.\\

\section*{Evne beskrivelse for Mestertyv}
\addcontentsline{toc}{section}{Evne beskrivelse for Mestertyv}

\subsection*{Dirke låse Niv. 3}
\addcontentsline{toc}{subsection}{Dirke låse Niv. 3}
Du kan nu Dirke låse niveau 3. Dette tager 15 minutter pr. lås.\\
\emph{Bemærk, at denne evne ikke gør at du kan åbne andre typer låse så som niveau 1 eller niveau 2.}\\

\subsection*{Hurtige fingre}
\addcontentsline{toc}{subsection}{Hurtige fingre}
jDu kan dirke låse hurtigere op. Det tager nu 2 min for en Niv. 1 lås, 4 min for en Niv. 2 lås og 6 min for en Niv. 3 lås.\\

\subsection*{Lange fingre}
\addcontentsline{toc}{subsection}{Lange fingre}
Mestertyven kan stjæle fra punge med låse på. Disse lommetyverier fungere på
præcis sammen måde som almindelig lommetyveri.\\

\subsection*{Lommetyveri Niv. 3}
\addcontentsline{toc}{subsection}{Lommetyveri Niv. 3}
Med Lommetyveri niveau 3 kan tyven lægge sin hånd på offerets pung i 5 sekunder, for derefter at stjæle alt (undtagen de sidste 3 mønter) i pungen. Offeret ved ikke, at de er blevet bestjålet. \\
Dette virker dog ikke, hvis offeret har en lås på sin pung.\\