\chapter*{Drikke}
\addcontentsline{toc}{chapter}{Drikke}

Med evnen Alkymi kan du lave drikke, der påvirker spillet og spillerne. De drikke, du kan lave med denne evne er ikke magiske, men kemiske.\\
Du kan altid kun blive påvirket af 1 positiv og 1 negativ bryg.\\
Drikke virker på 2 måder:
\begin{itemize}
    \item \textbf{Øjeblikkelig:} Disse drikke virker med de samme og effekten virker kun i det øjeblik de drikkes.
    \item \textbf{Over tid:} Disse drikke træder i kraft fra øjeblikket de indtages, og varer derefter over den tid, der beskrives på drikken.
\end{itemize}

Alle drikke holder indtil de er brugt, med få undtagelser. En drik kan derfor gemmes mellem spilgange, men vær dog opmærksom på at det er alkymistens ansvar, at indholdet ikke bliver for gammelt.\\
\textbf{\underline{Indholdet skal skiftes til hver scenarie!!}}

\section*{Hvordan man brygger}
\addcontentsline{toc}{section}{Hvordan man brygger}
Af hensyn til spillet skal du selv anskaffe glas, flasker og diverse redskaber, som er passende til den type alkymist du spiller. En alkymist uden redskaber kan ikke brygge nogle drikke.\\
Rollespilsforeningen A’kastin kan \textbf{ikke} udlåne glas/flasker/diverse der kan benyttes til alkymi.\\
Flaskerne må ikke være for store, da indholdet skal kunne drikkes i en mundfuld.\\
For at skabe så meget spil som muligt, skal du huske at din rolle skal fremstå ægte, og give andre spillere mulighed for at reagere på dit arbejde; derfor er det vigtigt at tage sig tid til brygningen. Det er ligeledes vigtigt, at du har alle ingredienserne der kræves til den specifikke drik.\\
\\\\\\
Når du har lavet din bryg, kan denne hældes på en flaske, hvorefter du kan sælge den. Den færdige drik skal kunne indtages og må derfor på ingen måde være sundhedsskadelig. Så selvom jord, bark, bær og andet fra skoven kan bruges til effekter i spillet, må drikken \textbf{ikke} indeholde dette. Drikken skal derfor indeholde en væske, såsom vand (gerne med frugtfarve eller saft), juice el. lign.\\
Det er alkymistens ansvar at sikre sig, at folk der indager drikken ikke er allergisk overfor ingredienserne i drikkene! Det er muligt at nogen er allergiske overfor nødder, mælk el. lign.\\
Hver drik der laves skal markeres med en seddel. Denne seddel skal påskrives drikkens navn, niveau og effekt. Effekten skal skrives på bagsiden, så det ikke er muligt for andre at læse drikkens effekt inden den indtages.\\

\emph{Eksempel: Forside: “Mortimors drik”. Bagside: “Dødens Essens. Du går på 0 LP”.}

\chapter*{Opskrifter}
\addcontentsline{toc}{chapter}{Opskrifter}
Alle opskrifter findes i dette regelsæt for din profession. Du skal kopiere opskrifterne over på et stykke in-game papir, som du kan have med i spillet. (In-game papir kan laves ved at dyppe almindeligt printerpapir i kaffe eller en stærk te for at give det farve)\\
Det er ikke tilladt at brygge efter hukommelsen, og det er ligeledes ikke tilladt at finde på sine egne drikke uden de relevante evner. Gøres dette, vil drikken være uden effekt.\\
\textbf{Fra Alkymi niveau 3:} Alkymisten opfordres til at eksperimentere og blande urter og ingredienser, som ikke nødvendigvis står i en opskrift. Hvis dette gøres, skal der efter drikken er laves, kontaktes en arrangør der vil fortælle dig drikkens effekt.
\begin{table}[H]
    \centering
    \begin{tabular}{|c|c|}
    \hline
    \rowcolor{cerulean!80}
        Niveau opskrift & Brygge tid \\\hline
        1 & 2 Minutter\\\hline
        2 & 4 Minutter \\\hline
        3 & 6 Minutter \\\hline
        4 & 8 Minutter \\\hline
    \end{tabular}
    \caption{Bryg tid}
    \label{tab:my_label}
\end{table}

\chapter*{Ingredienser}
\addcontentsline{toc}{chapter}{Ingredienser}
Hver opskrift kræver ingredienser. Ingredienserne i A'kastin vil altid bestå af en pose, der er markeret med ingrediensens navn. Når du har brugt en ingrediens afleveres den til en arrangør.\\
Der findes også ingredienser, som ikke gives ud af arrangørerne. Det kan for eksempel være øl fra kroen eller blod fra et offer. Uanset om ingrediensen er en væske eller et fast stof (for eksempel mos fra skoven), skal du bruge en beholder til at have ingrediensen i - meget gerne en beholder af glas. Vi gør opmærksom på, at alle drikke skal være drikkelige. Det vil sige ingen mos eller blade i dem.
\begin{center}
\begin{longtable}{|p{0.1\textwidth}|p{0.15\textwidth}|p{0.3\textwidth}|p{0.25\textwidth}|p{0.1\textwidth}|}
\hline
\rowcolor{cerulean!80}
Niveau & Navn & Beskrivelse & Effekt & Type \\\hline
\endhead

\hline \hline
\endlastfoot

1 & Blodbær &  Blodrøde bær. Dværgene kalder disse bær for Rustbær, da de ikke bliver lige så røde i Livet's ende. & Folk der spiser dem genvinder hurtigere kampkraft & Nyttig plante\\\hline
        
1 & Kongetand & En almindelig rod som mange koger suppe på. Normade stammer på De uendelige sletter bruger dem til at finde vand. & Denne rod hjælper med helbredelse. & Nyttig plante\\\hline

1 & Vorterod & En rod som findes under bøg træer. Den ligner en finger fyldt med vorter. & Folk der spiser den vil have mavepine. & Giftig plante \\\hline

2 & Ildgræs & Dette orange græs beklæder det meste af De Uendelige Sletter. & Dette græs kan antennes selvom det er vådt. & Nyttig plante \\\hline

2 & Kvælertang & Denne tang er kendt for at snøre sig omkring folk for at fortære deres krop under vandet. & Denne plante får blodet til at løbe langsommere & Giftig plante \\\hline

2 & Tudsebark & Denne bark samles fra sump træer i Mek og ligner en tudse set fra oven. & Denne bark fortynder blodet. & Giftig plante \\\hline

3 & Dværgerod & Denne plante menes at være i familie med Knoldselleri, denne art har dog et pragtfuld 'skæg' lavet af rødder. & Denne rod fortætter cellestrukturen. & Nyttig plante \\\hline

3 & Hybenholdt & Små spir fra en plante der normalt vokser omkring andre træer. & Denne plante er en kendt katalysator. & Nyttig plante \\\hline

3 & Slangefrø & Små frø fra en prægtig blomst. Når de rystes lyder de som en slange der hvæser. & Hvis disse frø spises skaber de illusioner. & Giftig plante \\\hline

4 & Cedertræ & Spåner fra et træ, som har været en stængel fra livets træ. & Te'en fra disse spåner har kraftige magiske og helbredende egenskaber. & Nyttig plante\\\hline

4 & Sort flue-svamp & En art af fluesvamp, som opstår når Ilsher blod rammer jorden. & Der findes ingen kendte arter som kan overleve at spise denne plante. & Giftig plante.\\\hline

\end{longtable}
\end{center}
