%  A simple AAU report template.
%  2015-05-08 v. 1.2.0
%  Copyright 2010-2015 by Jesper Kjær Nielsen <jkn@es.aau.dk>
%
%  This is free software: you can redistribute it and/or modify
%  it under the terms of the GNU General Public License as published by
%  the Free Software Foundation, either version 3 of the License, or
%  (at your option) any later version.
%
%  This is distributed in the hope that it will be useful,
%  but WITHOUT ANY WARRANTY; without even the implied warranty of
%  MERCHANTABILITY or FITNESS FOR A PARTICULAR PURPOSE.  See the
%  GNU General Public License for more details.
%
%  You can find the GNU General Public License at <http://www.gnu.org/licenses/>.
%
\documentclass[11pt,a4paper,openright]{report}
%%%%%%%%%%%%%%%%%%%%%%%%%%%%%%%%%%%%%%%%%%%%%%%%
% Language, Encoding and Fonts
% http://en.wikibooks.org/wiki/LaTeX/Internationalization
%%%%%%%%%%%%%%%%%%%%%%%%%%%%%%%%%%%%%%%%%%%%%%%%
% Select encoding of your inputs. Depends on
% your operating system and its default input
% encoding. Typically, you should use
%   Linux  : utf8 (most modern Linux distributions)
%            latin1 
%   Windows: ansinew
%            latin1 (works in most cases)
%   Mac    : applemac
% Notice that you can manually change the input
% encoding of your files by selecting "save as"
% an select the desired input encoding. 
\usepackage[utf8]{inputenc}
% Make latex understand and use the typographic
% rules of the language used in the document.
\usepackage[english, danish]{babel}
% Use the palatino font
\usepackage[sc]{mathpazo}
\linespread{1.05}         % Palatino needs more leading (space between lines)
% Choose the font encoding
\usepackage[T1]{fontenc}
%%%%%%%%%%%%%%%%%%%%%%%%%%%%%%%%%%%%%%%%%%%%%%%%
% Graphics and Tables
% http://en.wikibooks.org/wiki/LaTeX/Importing_Graphics
% http://en.wikibooks.org/wiki/LaTeX/Tables
% http://en.wikibooks.org/wiki/LaTeX/Colors
%%%%%%%%%%%%%%%%%%%%%%%%%%%%%%%%%%%%%%%%%%%%%%%%
% load a colour package
\usepackage[dvipsnames]{xcolor}
\definecolor{aaublue}{RGB}{33,26,82}% dark blue
% The standard graphics inclusion package
\usepackage{graphicx}
% Set up how figure and table captions are displayed
\usepackage{float}
\usepackage{caption}
\captionsetup{%
  font=footnotesize,% set font size to footnotesize
  labelfont=bf % bold label (e.g., Figure 3.2) font
}
% Make the standard latex tables look so much better
\usepackage{array,booktabs}
% Enable the use of frames around, e.g., theorems
% The framed package is used in the example environment
\usepackage{framed}
\usepackage{wrapfig}
\usepackage{multirow}

%%%%%%%%%%%%%%%%%%%%%%%%%%%%%%%%%%%%%%%%%%%%%%%%
% Mathematics
% http://en.wikibooks.org/wiki/LaTeX/Mathematics
%%%%%%%%%%%%%%%%%%%%%%%%%%%%%%%%%%%%%%%%%%%%%%%%
% Defines new environments such as equation,
% align and split 
\usepackage{amsmath}
% Adds new math symbols
\usepackage{amssymb}
% Use theorems in your document
% The ntheorem package is also used for the example environment
% When using thmmarks, amsmath must be an option as well. Otherwise \eqref doesn't work anymore.
\usepackage[framed,amsmath,thmmarks]{ntheorem}

%%%%%%%%%%%%%%%%%%%%%%%%%%%%%%%%%%%%%%%%%%%%%%%%
% Page Layout
% http://en.wikibooks.org/wiki/LaTeX/Page_Layout
%%%%%%%%%%%%%%%%%%%%%%%%%%%%%%%%%%%%%%%%%%%%%%%%
% Change margins, papersize, etc of the document
\usepackage[
  inner=30mm,% left margin on an odd page
  outer=30mm,% right margin on an odd page
  ]{geometry}
% Modify how \chapter, \section, etc. look
% The titlesec package is very configureable
\usepackage{titlesec}
\titleformat{\chapter}[display]{\normalfont\huge\bfseries}{\ }{20pt}{\Huge}
\titleformat*{\section}{\normalfont\Large\bfseries}
\titleformat*{\subsection}{\normalfont\large\bfseries}
\titleformat*{\subsubsection}{\normalfont\normalsize\bfseries}
\titleformat*{\paragraph}{\normalfont\normalsize\bfseries}
\titleformat*{\subparagraph}{\normalfont\normalsize\bfseries}

% Clear empty pages between chapters
\let\origdoublepage\cleardoublepage
\newcommand{\clearemptydoublepage}{%
  \clearpage
  {\pagestyle{empty}\origdoublepage}%
}
\let\cleardoublepage\clearemptydoublepage

% Change the headers and footers
\usepackage{fancyhdr}
\pagestyle{fancy}
\fancyhf{} %delete everything
\renewcommand{\headrulewidth}{0pt} %remove the horizontal line in the header
\fancyhead[R]{\small\nouppercase\leftmark} %even page - chapter title
\fancyhead[LO]{\small\nouppercase\rightmark} %uneven page - section title
\fancyhead[RO]{\thepage} %page number on all pages
\setlength{\headheight}{13.59999pt}
% Do not stretch the content of a page. Instead,
% insert white space at the bottom of the page
\raggedbottom
% Enable arithmetics with length. Useful when
% typesetting the layout.
\usepackage{calc}

%Paragraph spacing
\setlength{\parindent}{0em}
\setlength{\parskip}{1em}

%%%%%%%%%%%%%%%%%%%%%%%%%%%%%%%%%%%%%%%%%%%%%%%%
% Misc
%%%%%%%%%%%%%%%%%%%%%%%%%%%%%%%%%%%%%%%%%%%%%%%%
% Add bibliography and index to the table of
% contents
\usepackage[nottoc]{tocbibind}
% Add the command \pageref{LastPage} which refers to the
% page number of the last page
\usepackage{lastpage}
% Add todo notes in the margin of the document
\usepackage[
%  disable, %turn off todonotes
  colorinlistoftodos, %enable a coloured square in the list of todos
  textwidth=\marginparwidth, %set the width of the todonotes
  textsize=scriptsize, %size of the text in the todonotes
  ]{todonotes}
% include pdf files
\usepackage[final]{pdfpages}
% code
\usepackage{color}
\usepackage{listings}
\usepackage{tcolorbox}
\usepackage{subfig}
% coding colors defined %
\definecolor{codegreen}{rgb}{0,0.6,0}
\definecolor{codegray}{rgb}{0.5,0.5,0.5}
\definecolor{codepurple}{rgb}{0.58,0,0.82}
\definecolor{backcolour}{rgb}{0.95,0.95,0.92}

 
\lstdefinestyle{JavaStyle}{
    backgroundcolor=\color{backcolour},   
    commentstyle=\color{codegreen},
    keywordstyle=\color{magenta},
    numberstyle=\tiny\color{codegray},
    stringstyle=\color{codepurple},
    basicstyle=\footnotesize,
    breakatwhitespace=false,         
    breaklines=true,                 
    captionpos=b,
    frame=single,
    keepspaces=true,                 
    numbers=left,                    
    numbersep=5pt,                  
    showspaces=false,                
    showstringspaces=false,
    showtabs=false,
    language=Java,
    tabsize=4,
    extendedchars=true,
    literate={æ}{{\ae}}1 {Æ}{{\AE}}1 {ø}{{\o}}1 {Ø}{{\O}}1 {å}{{\r a}}1 {Å}{{\r A}}1,
}

\lstdefinelanguage{JavaScript}{
%alsoletter=æøå,
keywords={metode, liste, Dec, udskriv, hvis, tilføj, returner, hent, længde, somHeltal, Hel, imens, indsæt, ordbog, tekst, størrelse, nøgler},
keywordstyle=\color{blue}\bfseries,
ndkeywords={start},
ndkeywordstyle=\color{darkgray}\bfseries,
identifierstyle=\color{black},
sensitive=false,
comment=[l]{//},
morecomment=[s]{/*}{*/},
commentstyle=\color{purple}\ttfamily,
stringstyle=\color{red}\ttfamily,
morestring=[b]',
morestring=[b]"
}

\lstdefinestyle{JavaScriptStyle}{
    backgroundcolor=\color{backcolour},   
    commentstyle=\color{codegreen},
    keywordstyle=\color{magenta},
    numberstyle=\tiny\color{codegray},
    stringstyle=\color{codepurple},
    basicstyle=\footnotesize,
    breakatwhitespace=false,         
    breaklines=true,                 
    captionpos=b,
    frame=single,
    keepspaces=true,                 
    numbers=left,                    
    numbersep=5pt,                  
    showspaces=false,                
    showstringspaces=false,
    showtabs=false,
    language=javascript,
    tabsize=4,
    extendedchars=true,
    literate={æ}{{\ae}}1 {Æ}{{\AE}}1 {ø}{{\o}}1 {Ø}{{\O}}1 {å}{{\r a}}1 {Å}{{\r A}}1,
}
\lstset{style=JavaStyle}
\renewcommand{\lstlistingname}{Code}% Listing -> Code
\renewcommand{\lstlistlistingname}{List of \lstlistingname}% List of Listings -> List of Code
%%%%%%%%%%%%%%%%%%%%%%%%%%%%%%%%%%%%%%%%%%%%%%%%
% Hyperlinks
% http://en.wikibooks.org/wiki/LaTeX/Hyperlinks
%%%%%%%%%%%%%%%%%%%%%%%%%%%%%%%%%%%%%%%%%%%%%%%%
% Enable hyperlinks and insert info into the pdf
% file. Hypperref should be loaded as one of the 
% last packages

\usepackage{multirow}
\usepackage{csquotes}
\usepackage{chngpage}
\usepackage{pdflscape}
\usepackage{longtable}
\usepackage{makecell}

\usepackage[nobreak]{mdframed}

\numberwithin{equation}{chapter}

\usepackage{csvsimple}

\usepackage{tikz}
\usetikzlibrary{matrix}

\titlespacing*{\chapter}{0pt}{0pt}{40pt}
\usepackage{ulem}

\mdfdefinestyle{drikstyle}{%
linecolor=CornflowerBlue,linewidth=2pt,%
frametitlerule=true,%
frametitlebackgroundcolor=CornflowerBlue!20,
innertopmargin=\topskip,
}
\mdtheorem[style=drikstyle]{drik}{Drik}

\mdfdefinestyle{særligstyle}{%
linecolor=SpringGreen,linewidth=2pt,%
frametitlerule=true,%
frametitlebackgroundcolor=SpringGreen!20,
innertopmargin=\topskip,
}
\mdtheorem[style=særligstyle]{særlig}{Særlig}

\mdfdefinestyle{giftstyle}{%
linecolor=Bittersweet,linewidth=2pt,%
frametitlerule=true,%
frametitlebackgroundcolor=Bittersweet!20,
innertopmargin=\topskip,
}
\mdtheorem[style=giftstyle]{gift}{Gift}

\mdfdefinestyle{artefaktstyle}{%
linecolor=BlueGreen,linewidth=2pt,%
frametitlerule=true,%
frametitlebackgroundcolor=BlueGreen!20,
innertopmargin=\topskip,
}
\mdtheorem[style=artefaktstyle]{artefakt}{Artefakt}

\mdfdefinestyle{runerustningstyle}{%
linecolor=Emerald,linewidth=2pt,%
frametitlerule=true,%
frametitlebackgroundcolor=Emerald!20,
innertopmargin=\topskip,
}
\mdtheorem[style=runerustningstyle]{runerustning}{Runerustning}

\mdfdefinestyle{runevåbenstyle}{%
linecolor=RoyalBlue,linewidth=2pt,%
frametitlerule=true,%
frametitlebackgroundcolor=RoyalBlue!20,
innertopmargin=\topskip,
}
\mdtheorem[style=runevåbenstyle]{runevåben}{Runevåben}

\mdfdefinestyle{runeskjoldstyle}{%
linecolor=RoyalPurple,linewidth=2pt,%
frametitlerule=true,%
frametitlebackgroundcolor=RoyalPurple!20,
innertopmargin=\topskip,
}
\mdtheorem[style=runeskjoldstyle]{runeskjold}{Runeskjold}

\usepackage{tablefootnote}

\mdfdefinestyle{Meditationstyle}{%
linecolor=Emerald,linewidth=2pt,%
frametitlerule=true,%
frametitlebackgroundcolor=Emerald!20,
innertopmargin=\topskip,
}
\mdtheorem[style=Meditationstyle]{meditation}{Meditation}
\mdtheorem[style=Meditationstyle]{rite}{Rite}

\mdfdefinestyle{Orleksarvstyle}{%
linecolor=RedOrange,linewidth=2pt,%
frametitlerule=true,%
frametitlebackgroundcolor=RedOrange!20,
innertopmargin=\topskip,
}
\mdtheorem[style=Orleksarvstyle]{orleks arv}{Orleks arv}
\mdtheorem[style=Orleksarvstyle]{dHævn}{Dæmonisk hævn}

\mdfdefinestyle{Ritualstyle}{%
linecolor=Magenta,linewidth=2pt,%
frametitlerule=true,%
frametitlebackgroundcolor=Magenta!20,
innertopmargin=\topskip,
}
\mdtheorem[style=Ritualstyle]{ritual}{Ritual}

\mdfdefinestyle{Åndensgavestyle}{%
linecolor=RoyalBlue,linewidth=2pt,%
frametitlerule=true,%
frametitlebackgroundcolor=RoyalBlue!20,
innertopmargin=\topskip,
}
\mdtheorem[style=Åndensgavestyle]{åndens gave}{Åndernes gave}

\mdtheorem[style=Meditationstyle]{nBeskyt}{Naturens Beskyttelse}

\mdfdefinestyle{naturstyle}{%
linecolor=Goldenrod,linewidth=2pt,%
frametitlerule=true,%
frametitlebackgroundcolor=Goldenrod!20,
innertopmargin=\topskip,
}
\mdtheorem[style=naturstyle]{nly}{Naturens Ly}

\mdtheorem[style=drikstyle]{nvit}{Naturens Vitalitet}

\mdtheorem[style=særligstyle]{mkær}{Moder Naturs Kærlighed}

\mdtheorem[style=giftstyle]{nhævn}{Naturens hævn}

\mdtheorem[style=Åndensgavestyle]{nkaos}{Naturens Kaos}

\mdtheorem[style=runeskjoldstyle]{nbesk}{Naturens Beskyttelse}

\mdfdefinestyle{primærstyle}{%
linecolor=Plum,linewidth=2pt,%
frametitlerule=true,%
frametitlebackgroundcolor=Plum!20,
innertopmargin=\topskip,
}
\mdtheorem[style=primærstyle]{primærMagi}{Primær Magi}

\mdfdefinestyle{Lærdstyle}{%
linecolor=SkyBlue,linewidth=2pt,%
frametitlerule=true,%
frametitlebackgroundcolor=SkyBlue!20,
innertopmargin=\topskip,
}
\mdtheorem[style=Lærdstyle]{lærdMagi}{Den Lærdes Vej}

\mdfdefinestyle{ArkBanestyle}{%
linecolor=SeaGreen,linewidth=2pt,%
frametitlerule=true,%
frametitlebackgroundcolor=SeaGreen!20,
innertopmargin=\topskip,
}
\mdtheorem[style=ArkBanestyle]{arkBaneMagi}{Arkanaens Bane}

\mdfdefinestyle{magiMesterstyle}{%
linecolor=Orange,linewidth=2pt,%
frametitlerule=true,%
frametitlebackgroundcolor=Orange!20,
innertopmargin=\topskip,
}
\mdtheorem[style=magiMesterstyle]{mesterMagi}{Magiens Mester}

\mdfdefinestyle{sAritstyle}{%
linecolor=Melon,linewidth=2pt,%
frametitlerule=true,%
frametitlebackgroundcolor=Melon!20,
innertopmargin=\topskip,
}
\mdtheorem[style=sAritstyle]{sAritMagi}{Sfære Aritmetik}

\usepackage[T1]{fontenc} %thanks's daleif
\usepackage[utf8]{inputenc}
\usepackage[english, danish]{babel}
\newcommand{\tabitem}{~~\llap{\textbullet}~~}

\mdfdefinestyle{racestyle}{%
linecolor=PineGreen,linewidth=2pt,%
frametitlerule=true,%
frametitlebackgroundcolor=PineGreen!20,
innertopmargin=\topskip,
}
\mdtheorem[style=sAritstyle]{race}{Race Detaljer}

\mdfdefinestyle{sjælstyle}{%
linecolor=Cyan,linewidth=2pt,%
frametitlerule=true,%
frametitlebackgroundcolor=Cyan!20,
innertopmargin=\topskip,
}
\mdtheorem[style=sjælstyle]{sjæl}{Sælg din Sjæl}

\mdfdefinestyle{Korruptionstyle}{%
linecolor=PineGreen,linewidth=2pt,%
frametitlerule=true,%
frametitlebackgroundcolor=PineGreen!20,
innertopmargin=\topskip,
}
\mdtheorem[style=Korruptionstyle]{korruption}{Korruption}

\mdfdefinestyle{Faldenstyle}{%
linecolor=Tan,linewidth=2pt,%
frametitlerule=true,%
frametitlebackgroundcolor=Tan!20,
innertopmargin=\topskip,
}
\mdtheorem[style=Faldenstyle]{falden}{Falden Engel}

\mdfdefinestyle{Vandstyle}{%
linecolor=Aquamarine,linewidth=2pt,%
frametitlerule=true,%
frametitlebackgroundcolor=Aquamarine!20,
innertopmargin=\topskip,
}
\mdtheorem[style=Vandstyle]{vand}{Vand}

\mdfdefinestyle{Ildstyle}{%
linecolor=BrickRed,linewidth=2pt,%
frametitlerule=true,%
frametitlebackgroundcolor=BrickRed!20,
innertopmargin=\topskip,
}
\mdtheorem[style=Ildstyle]{ild}{Ild}

\mdfdefinestyle{Jordstyle}{%
linecolor=Sepia,linewidth=2pt,%
frametitlerule=true,%
frametitlebackgroundcolor=Sepia!20,
innertopmargin=\topskip,
}
\mdtheorem[style=Jordstyle]{jord}{Jord}

\mdfdefinestyle{Vindstyle}{%
linecolor=Gray,linewidth=2pt,%
frametitlerule=true,%
frametitlebackgroundcolor=Gray!20,
innertopmargin=\topskip,
}
\mdtheorem[style=Vindstyle]{vind}{Vind}

\mdtheorem[style=naturstyle]{passiv}{Passiv}

\mdtheorem[style=ArkBanestyle]{offensiv}{Offensiv}

\mdtheorem[style=runevåbenstyle]{defensiv}{Defensiv}

\mdtheorem[style=giftstyle]{kontrol}{Kontrol}

\mdtheorem[style=særligstyle]{zombie}{Zombie}

\mdtheorem[style=Orleksarvstyle]{nSjæl}{Sjæl}

\mdtheorem[style=Åndensgavestyle]{sygdom}{Sygdomens Mørke}

\mdtheorem[style=Ritualstyle]{død}{Død}

\mdtheorem[style=Ritualstyle]{prof}{Profession}